%
% Las margenes la codificacion y los Paquetes utilizados por el documento
%
\documentclass[12pt,letterpaper,oneside]{article}
\usepackage[utf8]{inputenc}
\usepackage[T1]{fontenc}
\usepackage[spanish]{babel}
\usepackage[right=4cm,left=2cm,top=3cm,bottom=2cm,headsep=0cm,footskip=0.5cm]{geometry}

%
% Pone el documento a doble espacio
%
\usepackage{setspace}
\doublespacing

%
% Indentacion de comienzo de parrafo:
% 3 lineas hacia abajo
% 2 letras hacia la derecha
%
\parindent 2em
\parskip 3em

\begin{document}

\section{INTRODUCCION}

Actualmente el proceso de desarrollo de software esta sufriendo grandes
cambios con las recientes apariciones de metodologías y técnicas
que modelan y definen los pasos a seguir en el momento de diseñar,
elaborar y mantener una aplicación o un paquete de software.

Muchas empresas invierten grandes cantidades de tiempo, dinero y
esfuerzo en la elaboración de procesos óptimos y claros que
permitan sincronizar y monitorear todo el proceso de desarrollo de
software ya que actualmente no hablamos de simples programas que
cumplen tareas simples, en cambio se desarrollan grandes aplicaciones
como sistemas operativos, suites de oficina, manejadores de bases de
datos, aplicaciones multimedia, etc. los cuales son difíciles de
controlar debido al gran numero de personas que trabajan en ellas, es
por esto que surgen interfaces sencillas que facilitan la integración
de los procesos mas comunes de programación, desarrollo y
mantenimiento de aplicaciones

\newpage\section{MARCO TEORICO}
\newpage\subsection{CONCEPTOS}
Reverse Proxy:
Un Proxy Inverso es un servidor proxy usado comunmente como front-end de multiples servidores web
y como es propio de un servidor proxy sin embargo de forma opuesta, retransmite las peticiones HTTP
desde cualquier host (dependiendo de la configuracion



--------------------------------------------------------------------------------------------------------------------------------------------------------------------------------
--------------------------------------------------------------------------------------------------------------------------------------------------------------------------------
--------------------------------------------------------------------------------------------------------------------------------------------------------------------------------s



Reverse proxy
From Wikipedia, the free encyclopedia
Jump to: navigation, search

A reverse proxy is a proxy server that is installed within the neighborhood of one or more servers. Typically, reverse proxies are utilized in front of webservers. All connections coming from the Internet addressed to one of the webservers are routed through the proxy server, which may either deal with the request itself or pass the request wholly or partially to the main webserver.

Contrast this with 'forward proxy', which is a proxy server configured in the end-user's browser.

There are several reasons for installing reverse proxy servers:

    * Security: the proxy server is an additional layer of defense and therefore protects the webservers further up the chain
    * Encryption / SSL acceleration: when secure websites are created, the SSL encryption is sometimes not done by the webserver itself, but by a reverse proxy that is equipped with SSL acceleration hardware.
    * Load distribution: the reverse proxy can distribute the load to several servers, each server serving its own application area. In the case of reverse proxying in the neighborhood of webservers, the reverse proxy may have to rewrite the URLs in each webpage (translation from externally known URLs to the internal locations).
    * Caching static content: A reverse proxy can offload the webservers by caching static content, such as images. Proxy caches of this sort can often satisfy a considerable amount of website requests, greatly reducing the load on the central web server.
    * Compression: the proxy server can optimize and compress the content to speed up the load time.
    * Spoon feeding: a dynamically generated page can be produced all at once and served to the reverse-proxy, which can then return it to the client a little bit at a time. The program that generates the page is not forced to remain open and tying up server resources during the possibly extended time the client requires to complete the file transfer.



http://en.wikipedia.org/wiki/Reverse_proxy
JSON
XML
Ruby
Ruby on Rails
Control de Versiones
WebDav
CGI-FCGI
\subsection{ESTADO DEL ARTE}

\section{IDE WEB}
\subsection{FORMULACION}
\subsubsection{CASOS DE USO}
\subsubsection{DEFINICION DE LA ARQUITECTURA}
\subsection{DISEÑO}
\subsubsection{MODELO CONCEPTUAL}
\subsubsection{DISEÑO DE INTERFAZ}
\subsubsection{DIAGRAMA DE PAQUETES}
\subsubsection{DIAGRAMA DE BASE DE DATOS}
\subsection{IMPLEMENTACION}
\subsubsection{IMPLEMENTACION DEL SOFTWARE}
\subsubsection{DOCUMENTACION INTENA DE CODIGO}
\subsubsection{MANUAL DE USUARIO}
\subsection{PRUEBAS}

\section{RESULTADOS}

\section{CONCLUSIONES Y RECOMENDACIONES}

\section{BIBLIOGRAFIA}


El desarrollo de esta aplicación web IDE (Entorno integrado de
desarrollo) abarca principalmente 2 temas que son derivados del nombre
general, estos son los IDEs y las aplicaciones web.

Un entorno integrado de desarrollo es un programa
compuesto por un conjunto de herramientas que, como su nombre lo dice,
integran y facilitan el trabajo de un programador.

Un IDE se compone principalmente por los siguientes subprogramas o
módulos:

\begin{enumerate}
\item Sistema de ordenamiento de archivos (código fuente, imágenes, utilidades, etc).
\item Editor de texto.
\item Compilador.
\item Depurador.
\end{enumerate}

Opcionalmente algunos IDEs incluyen:


\begin{enumerate}
\item Herramientas de automatización(compilación, depuración, ejecución, etc)
\item Sistema de control de versiones
\item Sistema de ayuda para la creación de Interfaces gráficas.
\end{enumerate}



Una aplicación web es aquella que los usuarios acceden a un servidor
web ubicado en Internet o en una Intranet por medio de un navegador
web.


\bigskip

Su fama se debe principalmente a la capacidad de centralización de la
información y a la facilidad \textmd{\textup{de acceso ya que implica
el uso de un navegador web y actualmente la mayoría de los sistemas
operativos incluyen por lo menos uno en su instalación por defecto.}}


\bigskip

{\mdseries\upshape
Algunos problemas que se destacan en las aplicaciones web incluyen
limitantes en la funcionalidad del cliente, ya que algunos métodos
comúnmente utilizadas en las interfaces gráficas o de escritorio
como dibujar en la pantalla o arrastrar y soltar no están soportadas
por los estándares web actuales, algunas soluciones propuestas
incluyen el uso de lenguajes interpretados en el lado del cliente (ej:
Javascript, flash, etc) que añaden cierta funcionalidad a las
interfaces, recientemente se han desarrollado tecnologías que
comunican estos lenguajes con aplicaciones del lado del servidor como
lo son PHP y AJAX, este ultimo es una técnica de desarrollo web que
permite al cliente cargar únicamente información adicional sin
necesidad de cargar toda la pagina otra vez.}


\bigskip

{\mdseries\upshape
Otro problema en el desarrollo de aplicaciones web son (en algunos
casos) los limitados anchos de banda, algunas soluciones incluyen la
compresión de datos, disminución del contenido, o la inclusión de
AJAX que como se menciona anteriormente permite enviar peticiones al
servidor web para obtener únicamente la información necesaria y
usando Javascript en el cliente para procesar la respuesta del servidor
web, en este caso la carga de la pagina inicial es mas lenta debido a
que se tiene que descargar todo el código Javascript.}


\bigskip


\bigskip


\bigskip


\bigskip


\bigskip


\bigskip


\bigskip


\bigskip


\bigskip


\bigskip


\bigskip


\bigskip


\bigskip


\bigskip


\bigskip


\bigskip


\bigskip


\bigskip

\section{2. PLANTEAMIENTO Y FORMULACI\'ON DEL PROBLEMA}

\bigskip

\subsection[2.1 \ El problema de investigación]{2.1 \ El problema de
investigación}

\bigskip

{\mdseries\upshape
El problema nace a partir de varias necesidades, lo primero que se
intenta resolver es la inexistencia de un sistema de desarrollo
disponible a través de Internet o de una Intranet y que se pueda
acceder a el utilizando únicamente un navegador web, disminuyendo
así la distribución e instalación de software adicional en miles
de clientes y con la ventaja de que se pueda utilizar en cualquier
cliente independientemente de la versión del sistema operativo que
tengan instalado.}


\bigskip

Otro problema que se intenta resolver es la descentralización del
código fuente y demás archivos, que a su vez genera problemas
comunes como la perdida o el desorden de la información o la
dificultad de controlar adecuadamente las versiones actuales de los
archivos.


\bigskip

\subsection{2.2 Estado de arte, antecedentes científicos y
tecnológicos en empresas o medios}

\bigskip

En los últimos días las aplicaciones web se han visto afectadas por
la implementación de una técnica conocida como AJAX, la cual
añade la habilidad de cargar de forma dinámica partes de la
interfaz, en este momento podemos ver aplicaciones muy útiles y
fáciles de usar, las cuales accedemos totalmente utilizando
únicamente el navegador, entre estas se destacan:


\bigskip


\begin{enumerate}
\item {\selectlanguage{spanish}
\textbf{Writely:} Procesador de textos, con acceso mediante el
navegador, posee un interfaz ajax y varias características
adicionales como la posibilidad de añadir etiquetas (tags) a los
documentos y la de compartir (ya sea en lectura o también en
escritura) con otros usuarios.}


\bigskip
\item \textbf{Gmail:} es un servicio de correo electrónico gratuito en
etapa de pruebas (beta), que ha captado la atención de los medios de
información por sus innovaciones tecnológicas, su capacidad y por
algunas quejas de violación a la privacidad de los usuarios.


\bigskip
\item \textbf{Google Calendar: Gratuito servicio de calendarios en linea,
permite mantener presente facilmente fechas de cumpleaños, reuniones,
busqueda en internet de eventos importantes, etc.}


\bigskip
\end{enumerate}

\begin{enumerate}
\item \textbf{Flickr:} es un sitio web de organización de
fotografías digitales y red social. Fue desarrollado por Ludicorp,
una empresa de Vancouver, Canadá, fundada en 2002. En marzo de 2005,
Flickr y Ludicorp fueron compradas por Yahoo!. El servicio es utilizado
extensamente por bloggers como depósito de fotos. El sistema de
Flickr permite hacer búsquedas de imágenes por etiquetas (tags),
por fecha y por licencias de Creative Commons.
\end{enumerate}

\bigskip

En el mundo de los IDEs encontramos herramientas que han cambiado la
forma en la que se diseñan y desarrollan los programas de
computadora, los mas importantes actualmente son:


\bigskip


\begin{enumerate}
\item \textbf{Eclipse:} Eclipse es una IDE multiplataforma libre para
crear aplicaciones clientes de cualquier tipo. La primera y más
importante aplicación que ha sido realizada con este entorno es la
afamado IDE Java llamado Java Development Toolkit (JDT) y el compilador
incluido en Eclipse, que se usaron para desarrollar el propio Eclipse.


\bigskip
\item \textbf{Kdevelop:} es un entorno integrado de desarrollo con
licencia GPL para sistemas Linux y otros sistemas Unix, a diferencia de
muchas otras interfaces de desarrollo, KDevelop no cuenta con un
compilador propio, por lo que depende de gcc para producir código
binario. Su última versión se encuentra actualmente bajo desarrollo
y soporta entre otros lenguajes de programación a C, C++, Java, SQL,
Python, Perl, Pascal y Bash.


\bigskip
\item \textbf{Visual Studio .NET:} es un conjunto de herramientas
integrado para la construcción y desarrollo de servicios web XML y
soluciones Web creado por Microsoft y ampliamente utilizado en el
desarrollo de aplicaciones basadas en Windows.
\end{enumerate}

\bigskip

La aplicación que intentamos desarrollar no tiene antecedentes
registrados, sin embargo, \ encontramos un conjunto de aplicaciones que
se aproximan un poco a la idea central con la diferencia de que se
limitan a ser una simple interfaz de arrastre{}-y{}-suelte (drag and
drop) para el desarrollo de paginas web, entre ellas la mas importante
quizá sea Google Page Creator, esta es una
herramienta en linea gratuita que permite a cada persona crear y
publicar utiles y atractivas paginas web en cuestion de minutos, el
hosting corre por cuenta de Google el cual hospeda automáticamente
las paginas en la siguiente dirección:
http://tucuenta.googlepages.com.


\bigskip


\bigskip


\bigskip


\bigskip


\bigskip


\bigskip


\bigskip


\bigskip


\bigskip


\bigskip


\bigskip


\bigskip


\bigskip


\bigskip

\section{3. OBJETIVOS}

\bigskip

\subsection[3.1 \ General]{3.1 \ General}

\bigskip
Desarrollar aplicación web IDE (Integrated Development Environment)
basado en el concepto RIA(Rich Internet Application).


\bigskip

\subsection[3.2 \ Específicos]{3.2 \ Específicos}

\bigskip
Investigar un método eficiente de comunicación con el cliente, que
en su mayor parte se utilizara para el envío de archivos ejecutables.

\bigskip


\begin{enumerate}
\item Seleccionar el lenguaje de programación y las herramientas para
las que el IDE estará enfocado.
\end{enumerate}

\bigskip


\begin{enumerate}
\item Investigar acerca de los diferentes lenguajes y técnicas de
desarrollo de aplicaciones RIA y seleccionar la mas conveniente para el
desarrollo de esta aplicación.


\bigskip
\item Diseñar e implementar una interfaz fácil de utilizar que
permita la edición, compilación y ejecución de programas.
\end{enumerate}

\bigskip


\begin{enumerate}
\item Diseñar e implementar un modulo que permita administrar y
manejar las versiones de la información de un proyecto de
programación.
\end{enumerate}

\bigskip

\subsection[3.3 \ Estratégicos]{3.3 \ Estratégicos}

\bigskip
Desarrollar una aplicación que facilite el trabajo de un programador.

\bigskip


\begin{enumerate}
\item Desarrollar una aplicación que permita administrar las
configuraciones y versiones de forma transparente al usuario.
\end{enumerate}

\bigskip


\begin{enumerate}
\item Desarrollar una aplicación a la que se pueda acceder utilizando
únicamente un navegador web.
\end{enumerate}

\bigskip


\bigskip


\bigskip


\bigskip


\bigskip


\bigskip


\bigskip


\bigskip


\bigskip

\section{4. JUSTIFICACION}

\bigskip

\subsection[4.1 \ Importancia y significado]{4.1 \ Importancia y
significado}

\bigskip

El problema visto desde un marco mas general recae en la necesidad
actual de disponer de forma centralizada y sin dependencias adicionales
de software o hardware de las aplicaciones comúnmente utilizadas en
ambientes de escritorio, como lo son lectores de feeds, compresores y
descompresores, suites de oficina, herramientas de cifrado, etc.


\bigskip

Adicional a esto al llevar a cabo este proyecto, se dará solución a
varios problemas, el primer problema esta, en que como cada maquina
debe tener su propio IDE instalado y configurado según las
políticas del grupo de desarrollo, cualquier pequeño cambio en
estas políticas, conlleva a volver a configurar cada maquina en donde
este el IDE. ?`Como centralizar el control sobre el IDE de una
organización desarrolladora para minimizar el trabajo de
mantenimiento sobre éste?.

\bigskip

El segundo problema proviene del hardware y del software disponible en
la organización desarrolladora. Al tener un IDE basado en Web, cada
maquina solo tiene que tener un navegador Web para empezar a funcionar,
muy ligado al primer inconveniente, independizar el entorno de
desarrollo de las especificaciones hardware de la maquina cliente y
hasta del sistema operativo en el que se ejecute es una gran ventaja
para los desarrolladores.


\bigskip

El tercer problema recae en la descentralización del sitio de
almacenamiento de la información, esto genera problemas comunes como
la perdida o el desorden de dicha información convirtiendo una
actividad tan simple como guardar un archivo en un problema ligado a la
la sincronización de las diferentes fuentes de dicho archivo.


\bigskip

\subsection{4.2 Beneficios que traerá, desarrollo para la
institución, conocimiento o personal}

\bigskip


\begin{enumerate}
\item El control de versiones es una herramienta básica para el
desarrollo de aplicaciones en grupo. El IDE basado en Web obviaría la
necesidad de instalar clientes de control de versiones en cada maquina,
pues el código estaría bajo un control de versiones centralizado y
el manejo de cambios seria en su mayor parte transparente para el
usuario.


\bigskip
\item Se disminuye considerablemente la instalación y mantenimiento de
las herramientas de programación comúnmente utilizadas como
compiladores, editores, etc.


\bigskip
\item El proceso de compilación puede centralizarse, requiriendo
únicamente un servidor de buen rendimiento y contando incluso con
pobres instalaciones o equipos con bajas capacidades de hardware que
pasaran a ser clientes de la aplicación.
\end{enumerate}

\bigskip


\begin{enumerate}
\item
A nivel personal, el desarrollo de este proyecto permitirá afianzar
los conocimientos obtenidos durante la carrera, en gran manera en el
área de diseño de interfaces y desarrollo de aplicaciones web.
\end{enumerate}

\bigskip

\subsection{4.3 Impacto}

\bigskip

{\bfseries\itshape
4.3.1 \ Impacto Económico}


\bigskip

{\mdseries\upshape
Permitirá a una empresa de desarrollo de software ahorrar gastos en la
compra de herramientas de programación y en la compra de costosos
equipos para los programadores, permitiendo la transformación de
equipos antiguos en útiles herramientas de trabajo.}


\bigskip

{\bfseries\itshape
4.3.2 \ Impacto Social}


\bigskip

El sitio geográfico de trabajo ya no seria un problema, pues como el
acceso al IDE seria vía Web, no habría diferencia en trabajar desde
cualquier PC conectado a Internet.


\bigskip


\bigskip

{\bfseries\itshape
4.3.3 Impacto Científico y Tecnológico}


\bigskip

{\mdseries\upshape
Las herramientas de control de versiones quedaran obsoletas permitiendo
al usuario deshacer y rehacer los cambios efectuados durante el proceso
de edición de la información.}


\bigskip


\bigskip


\bigskip


\bigskip


\bigskip


\bigskip


\bigskip


\bigskip


\bigskip


\bigskip


\bigskip


\bigskip


\bigskip


\bigskip


\bigskip


\bigskip


\bigskip


\bigskip


\bigskip

\section[5. \ MARCO DE REFERENCIA]{5. \ MARCO DE REFERENCIA}

\bigskip

\subsection[5.1 \ Marco teórico]{5.1 \ Marco teórico}

\bigskip

A continuación se describirán los aspectos teóricos necesarios
para la realización de este proyecto:


\bigskip

{\bfseries
JavaScript}

Es un lenguaje interpretado orientado a las páginas web, con una
sintaxis semejante a la del lenguaje Java.

El lenguaje fue inventado por Brendan Eich en la empresa Netscape
Communications, que es la que fabricó los primeros navegadores de
Internet comerciales.

Apareció por primera vez en el producto de Netscape llamado Netscape
Navigator 2.0.

Tradicionalmente, se venía utilizando en páginas web HTML, para
realizar tareas y operaciones en el marco de la aplicación cliente
servidor. Con la irrupción de Web 2.0, JavaScript se ha convertido en
un verdadero lenguaje de programación que aporta la potencia de
cálculo del navegador para aumentar la usabilidad de aplicaciones Web
con técnicas avanzadas como AJAX o JCC.


\bigskip

{\bfseries
PHP}

Es un lenguaje de programación usado generalmente para la creación
de contenido para sitios web. PHP es el acrónimo recursivo de
{\textquotedbl}PHP: Hypertext Preprocessor{\textquotedbl}, inicialmente
PHP Tools, o, Personal Home Page Tools, es un lenguaje interpretado
usado para la creación de aplicaciones para servidores, o creación
de contenido dinámico para sitios web, y últimamente también para
la creación de otro tipo de programas incluyendo aplicaciones con
interfaz gráfica usando la librería GTK+.


\bigskip

{\bfseries
XHTML}

Acrónimo inglés de eXtensible Hypertext Markup Language (lenguaje
extensible de marcado de hipertexto), es el lenguaje de marcado pensado
para sustituir a HTML como estándar para las páginas web. XHTML es
la versión XML de HTML, por lo que tiene, básicamente, las mismas
funcionalidades, pero cumple las especificaciones, más estrictas, de
XML. Su objetivo es avanzar en el proyecto del World Wide Web
Consortium de lograr una web semántica, donde la información, y la
forma de presentarla estén claramente separadas. En este sentido,
XHTML serviría únicamente para transmitir la información que
contiene un documento, dejando para hojas de estilo (como las hojas de
estilo en cascada) y JavaScript su aspecto y diseño en distintos
medios (ordenadores, PDAs, teléfonos móviles, impresoras, etc).


\bigskip


\bigskip

{\bfseries
MySQL}

\textmd{E}s uno de los Sistemas Gestores de bases de Datos (SQL) más
populares desarrolladas bajo la filosofía de código abierto.

La desarrolla y mantiene la empresa MySQL AB pero puede utilizarse
gratuitamente y su código fuente está disponible.


\bigskip

MySQL es un sistema de administración de bases de datos. Una base de
datos es una colección estructurada de datos. Esta puede ser desde
una simple lista de compras a una galería de pinturas o el vasto
monto de información en un red corporativa. Para agregar, accesar y
procesar datos guardados en un computador, usted necesita un
administrador como MySQL Server. Dado que los computadores son muy
buenos manejando grandes cantidades de información, los
administradores de bases de datos juegan un papel central en
computación, como aplicaciones independientes o como parte de otras
aplicaciones.


\bigskip

MySQL es un sistema de administración relacional de bases de datos.
Una base de datos relacional archiva datos en tablas separadas en vez
de colocar todos los datos en un gran archivo. Esto permite velocidad y
flexibilidad. Las tablas están conectadas por relaciones definidas
que hacen posible combinar datos de diferentes tablas sobre pedido.


\bigskip


\bigskip


\bigskip

\subsection[5.2 \ Marco conceptual]{5.2 \ Marco conceptual}

\bigskip

{\bfseries
Sistema de Control de Versiones}

Un Sistema de Control de Versiones permite gestionar las versiones de
todos los ítems de configuración que forman la línea base de un
producto o una configuración del mismo. Este tipo de sistemas
facilitan la administración de las distintas versiones de cada
producto desarrollado junto a las posibles especializaciones realizadas
para algún cliente específico.

Los sistemas de control son utilizados principalmente en la industria
del software para controlar las distintas versiones del código
fuente. Sin embargo, los mismos conceptos son aplicables en otros
ámbitos y no solo para código fuente sino para documentos,
imágenes...

Aunque un sistema de control de versiones puede realizarse de forma
manual, es muy aconsejable disponer de herramientas que faciliten esta
gestión (CVS, Subversion, Source Safe, Clear Case, Darcs, etc)


\bigskip

{\bfseries
Base de Datos}

Es un conjunto de datos que pertenecen al mismo contexto almacenados
sistemáticamente para su uso posterior. En este sentido, una
biblioteca puede considerarse una base de datos compuesta en su
mayoría por documentos y textos impresos en papel e indexados para su
consulta.


\bigskip

En la actualidad, y gracias al desarrollo tecnológico de campos como
la informática y la electrónica, la mayoría de las bases de datos
tienen formato electrónico, que ofrece un amplio rango de soluciones
al problema de almacenar datos.

En informática existen los sistemas gestores de bases de datos (SGBD),
que permiten almacenar y posteriormente acceder a los datos de forma
rápida y estructurada. Las propiedades de los sistemas gestores de
bases de datos se estudian en informática.

Las aplicaciones más usuales son para la gestión de empresas e
instituciones públicas. También son ampliamente utilizadas en
entornos científicos con el objeto de almacenar la información
experimental.


\bigskip

Aunque las bases de datos pueden contener muchos tipos de datos, algunos
de ellos se encuentran protegidos por las leyes de varios países. Por
ejemplo en España, los datos personales se encuentran protegidos por
la Ley Orgánica de Protección de Datos de Carácter Personal
(LOPD).


\bigskip

{\bfseries
Aplicación Web}

\textmd{E}n ingeniería de software una aplicación web es aquella que
los usuarios usan accediendo a un servidor web a través de Internet o
de una intranet. Las aplicaciones web son populares debido a la
practicidad del navegador web como cliente ligero. La habilidad para
actualizar y mantener aplicaciones web sin distribuir e instalar
software en miles de potenciales clientes es otra razón de su
popularidad. Aplicaciones como los webmails, wikis, weblogs, MMORPGs,
tiendas en linea son ejemplos bien conocidos de aplicaciones web.

{\bfseries
RIA}

Rich Internet Applications o Aplicaciones Ricas de Internet son
aplicaciones web que tienen funcionalidades propias de las
tradicionales aplicaciones de escritorio.

Usualmente las RIAs transfieren el procesamiento necesario para la
interfaz del usuario al cliente web pero mantienen la parte logica y de
procesamiento de los datos en el servidor de la aplicación.

Generalmente las RIAs:


\bigskip


\begin{enumerate}
\item Son ejecutadas en un navegador web o no necesitan instalación
adicional de software.
\item Son ejecutadas localmente en un entorno seguro llamado SandBox
(caja de arena).
\end{enumerate}

\bigskip

Los métodos o técnicas utilizadas para desarrollar RIAs son:


\begin{enumerate}
\item JavaScript
\item Macromedia Flash
\item Controles ActiveX
\item Applets de Java
\item Lenguajes de Interfaz de Usuario (ej: XUL).
\end{enumerate}

\bigskip


\bigskip


\bigskip

\subsection{5.3 \ Marco contextual}

\bigskip

Al tratarse de un proyecto de grado, se debe realizar en un ambiente
académico, en el cual espero tener la colaboración de distintos
profesores y compañeros de estudio que han facilitado mi proceso de
formación como ingeniero de sistemas. Debido a que la herramienta es
diseñada para programadores se espera también escuchar sus
opiniones, concejos y posibles mejoras en el sistema. Al tratarse de
una aplicación Web permitirá el acceso a los usuario desde
cualquier lugar que cuente con una conexión a Internet.


\bigskip

\subsection{5.4 Tipo de proyecto: Teórico y Práctico}

\bigskip

Debido a que el problema que se intenta resolver es muy especifico el
proyecto a realizar es de tipo práctico, una aplicación web, y
teórico ya que algunos conceptos no son claros o no están del todo
bien definidos y requieren de un previo proceso de investigación que
faciliten la toma de decisiones en la realización del proyecto, estos
son: la técnica de desarrollo de aplicaciones RIA a utilizar y la
solución al problema del envío de archivos a la maquina cliente.


\bigskip


\bigskip


\bigskip


\bigskip


\bigskip


\bigskip

\section{6. ASPECTOS METODOL\'OGICOS}

\bigskip

\subsection{6.1 Definición y delimitación del problema}

\bigskip

Un IDE puede convertirse en una aplicación muy compleja que integre
muchas herramientas de diversos lenguajes de programación, debido a
esto la aplicación únicamente soportará las herramientas
(señalador de sintaxis en el editor, distribución del contenido del
proyecto, compilador, sistema de ejecución) de un lenguaje de
programación que sera seleccionado en el transcurso del proyecto, a
pesar de esto, será diseñado de forma que sea fácil extenderlo a
otras herramientas y lenguajes de programación.


\bigskip

Algunos IDEs incluyen sus propios compiladores y depuradores como es el
caso del compilador Eclipse, para resolver el problema de la
compilación se crearan interfaces de comunicación con los
compiladores existentes (ej: gcc, javac, etc)


\bigskip

El proceso de ejecución sera de forma local, lo que significa que el
programa no sera ejecutado en el lado del servidor sino que sera
transferido al cliente y este se encargara de ejecutarlo.


\bigskip

\subsection[6.2 \ Resultados esperados]{6.2 \ Resultados esperados}

\bigskip

Se espera como resultado una aplicación Web basada en el concepto RIA
que centralice y facilite el proceso de programación y que a su vez
integre el sistema de control de versiones de manera eficaz e intuitiva
para el usuario.


\bigskip

\subsection{6.3 Estrategias de desarrollo}

\bigskip

Para el desarrollo de este
proyecto se seguirá un proceso de desarrollo que agilice el proceso
de desarrollo disminuyendo la documentación sin perder claridad sobre
el diseño de la aplicación. Para apoyar dicho proceso, se usarán
herramientas muy útiles para la realización de documentación
como UML.


\bigskip

\subsection{6.4 Plan de actividades}

\bigskip

Para alcanzar los objetivos de este proyecto se propone seguir la
siguiente metodología:


\bigskip


\begin{enumerate}
\item Realizar consultas periódicamente con el profesor que dirige la
tesis \ con el fin de aclarar dudas y \ afianzar conceptos que se
aplicarán en el desarrollo del proyecto.


\bigskip
\end{enumerate}

\begin{enumerate}
\item Paralelo a la implementación se debe realizar un proceso de
dominio de tecnologías \ que se utilizarán para la implementación
del proyecto. Al finalizar el proyecto se espera dominar correctamente
las siguientes tecnologías: PHP, MySQL, Javascript, \ XHTML.


\bigskip
\end{enumerate}

\begin{enumerate}
\item Realizar un plan de pruebas que permita mejorar la calidad del
proyecto.


\bigskip
\item Consultar con usuarios de pruebas acerca de mejoras con respecto a
la interfaz, eliminando ambig\"uedades y confusiones en el momento de
ejecución de la aplicación.
\end{enumerate}

\bigskip


\begin{enumerate}
\item Hacer un proceso periódico de documentación teniendo así una
visión clara del estado del proyecto con relación al cronograma de
actividades.
\end{enumerate}

\bigskip


\bigskip


\bigskip


\bigskip


\bigskip


\bigskip


\bigskip


\bigskip


\bigskip


\bigskip


\bigskip

\section{7. ASPECTOS ADMINISTRATIVOS}

\bigskip

\subsection{7.1 Personal disponible y solicitado}

\bigskip

Inicialmente se requiere la ayuda y coordinación del director del
proyecto de grado, una vez terminada la implementación se requieren
personas ajenas al desarrollo de la aplicación que se encarguen de
realizar pruebas al sistema, y que a su vez aconsejen y propongan
mejoras en la interfaz del usuario.

Principalmente también se necesita el trabajo del desarrollador del
proyecto que \ es quien lo va a llevar a cabo.


\bigskip

\subsection{7.2 Infraestructura física disponible y solicitada}

\bigskip

Se requiere un ambiente de trabajo que facilite la concentración y el
buen desempeño, además se requiere de un equipo de buen rendimiento
que permita la ejecución de herramientas como: GNU/Linux, Apache2,
PHP, MySQL, DIA, Mozilla Firefox.


\bigskip


\bigskip


\bigskip


\bigskip


\bigskip

\section[8. \ TABLA DE CONTENIDO]{8. \ TABLA DE CONTENIDO}

\bigskip


\begin{enumerate}
\item INTRODUCCION
\item MARCO TEORICO

\begin{enumerate}
\item CONCEPTOS
\item ESTADO DEL ARTE
\end{enumerate}
\item IDE WEB

\begin{enumerate}
\item FORMULACION

\begin{enumerate}
\item CASOS DE USO
\item DEFINICION DE LA ARQUITECTURA
\end{enumerate}
\item DISEÑO

\begin{enumerate}
\item MODELO CONCEPTUAL
\item DISEÑO DE INTERFAZ
\item DIAGRAMA DE PAQUETES
\item DIAGRAMA DE BASE DE DATOS
\end{enumerate}
\item IMPLEMENTACION

\begin{enumerate}
\item IMPLEMENTACION DEL SOFTWARE
\item DOCUMENTACION INTERNA DE C\'ODIGO
\item MANUAL DE USUARIO
\end{enumerate}
\item PRUEBAS
\end{enumerate}
\item RESULTADOS
\item CONCLUSIONES Y RECOMENDACIONES
\item BIBLIOGRAFIA
\end{enumerate}

\bigskip


\bigskip


\bigskip


\bigskip


\bigskip


\bigskip


\bigskip


\bigskip


\bigskip


\bigskip


\bigskip


\bigskip


\bigskip


\bigskip


\bigskip


\bigskip


\bigskip


\bigskip


\bigskip


\bigskip

\section{9. CRONOGRAMA}

\bigskip



\begin{center}
[Warning: Image not found]
\end{center}

\bigskip


\bigskip


\bigskip


\bigskip


\bigskip


\bigskip


\bigskip


\bigskip


\bigskip

\section{10. BIBLIOGRAF\'IA}

\bigskip

[wpscv] Wikipedia: Sistema de Control de Versión, disponible en:

``http://es.wikipedia.org/wiki/Sistema\_de\_control\_de\_versión'', 21
mayo 2006.


\bigskip

[wpjs] Wikipedia: JavaScript, disponible en:

``http://es.wikipedia.org/wiki/JavaScript'', 13 junio 2006


\bigskip

[wpxhtml] Wikipedia: XHTML, disponible en:

{\textquotedbl}http://es.wikipedia.org/wiki/XHTML{\textquotedbl}, 13
junio 2006


\bigskip

[wpphp] Wikipedia: PHP, disponible en:

{\textquotedbl}http://es.wikipedia.org/wiki/PHP{\textquotedbl}, 13 junio
2006


\bigskip

[wpbd] Wikipedia: Base de Datos, disponible en:

{\textquotedbl}http://es.wikipedia.org/wiki/Base\_de\_datos{\textquotedbl},
13 junio 2006


\bigskip

[wpaw] Wikipedia: Aplicación web, disponible en:

{\textquotedbl}http://es.wikipedia.org/wiki/Aplicación\_web{\textquotedbl},
13 junio 2006

[wpria] Wikipedia: Rich Internet Application, disponible en:

{\textquotedbl}http://en.wikipedia.org/wiki/Rich\_Internet\_Application{\textquotedbl},
13 junio 2006


\bigskip

[w3crwc] W3C: Rich Web Clients, disponible en:

``http://www.w3.org/2006/rwc/'', 13 junio 2006
\end{document}
