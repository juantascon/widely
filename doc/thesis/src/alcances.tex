\section{ALCANCES}

\subsection{Definición y delimitación del problema}

Un IDE puede convertirse en una aplicación muy compleja que integre muchas herramientas de diversos lenguajes de programación, debido a esto la aplicación desarrollada únicamente soporta las herramientas del lenguaje de programación python, a pesar de esto, su diseño y arquitectura permiten que sea fácil extenderlo a otras herramientas y lenguajes de programación.

Algunos IDEs incluyen sus propios compiladores y depuradores como es el caso del IDE Eclipse, para resolver el problema de la compilación se creó una interfaz de comunicación con el compilador que se incluye en la distribución estandar de python.

El proceso de ejecución se realiza de forma local, lo que significa que los programas generados (los proyectos de desarrollo que realizan los usuarios) no son ejecutado en el lado del servidor sino que son transferidos al cliente y este se encarga de su ejecucion.

\subsection{Estrategias de desarrollo}

Para el desarrollo de este proyecto se siguió un proceso de desarrollo que permitió agilizar los procesos de formulación y diseño de software, disminuyendo la documentación, pero, sin perder claridad sobre el entorno y la funcionalidad interna de la aplicación.

Para apoyar dicho proceso, se usaron herramientas muy útiles para la realización de documentación como son:

* El lenguaje de modelado UML ( http://www.ibm.com/software/rational/uml/ )
* La herramienta de libre distribucion DIA ( http://www.gnome.org/projects/dia/ )
* El editor de graficos vectoriales de codigo abierto Inkscape ( www.inkscape.org )
* El sistema operativo de libre distribucion ArchLinux ( www.archlinux.org )
* El sistema manejador de contenido de codigo abierto Radiant ( www.radiantcms.org )
