\section{INTRODUCCION}

Actualmente el proceso de desarrollo de software esta sufriendo grandes cambios con las recientes apariciones de metodologías y técnicas que modelan y definen los pasos a seguir en el momento de diseñar, elaborar y mantener una aplicación o un paquete de software.

Muchas empresas invierten grandes cantidades de tiempo, dinero y esfuerzo en la elaboración de procesos óptimos y claros que permitan sincronizar y monitorear todo el proceso de desarrollo de software ya que actualmente no hablamos de simples programas que cumplen tareas simples, en cambio se desarrollan grandes aplicaciones como sistemas operativos, suites de oficina, manejadores de bases de datos, aplicaciones multimedia, etc. los cuales son difíciles de controlar debido al gran numero de personas que trabajan en ellas, es por esto que surgen interfaces sencillas que facilitan la integración de los procesos mas comunes de programación, desarrollo y mantenimiento de aplicaciones
