\section{JUSTIFICACIÓN}


\subsection{Importancia y significado}

El problema, visto desde un marco mas general, recae en la necesidad actual de disponer de las aplicaciones comúnmente utilizadas en ambientes de escritorio, como lo son lectores de feeds, compresores y descompresores, suites de oficina, herramientas de cifrado, etc. de forma centralizada y sin dependencias adicionales de software o hardware.

Adicional a esto, al llevar a cabo este proyecto, se dará solución a varios problemas, el primer problema está en como cada máquina debe tener su propio IDE instalado y configurado según las políticas del grupo de desarrollo, cualquier pequeño cambio en estas políticas, obliga a reconfigurar cada máquina en donde esté el IDE. La herramienta desarrollada permite centralizar el control que tiene una organización sobre el IDE, minimizando el trabajo de mantenimiento de los clientes.

El segundo problema proviene del hardware y del software disponible en la organización desarrolladora. Al tener un IDE basado en Web, cada máquina solo necesita un navegador Web para empezar a funcionar, muy ligado al primer inconveniente, por esto, es una gran ventaja para los desarrolladores independizar el entorno de desarrollo de las especificaciones hardware de la máquina cliente y hasta del sistema operativo en el que se ejecute.

El tercer problema recae en la descentralización del sitio de almacenamiento de la información, esto genera problemas comunes como la perdida o el desorden de dicha información convirtiendo una actividad tan simple como guardar un archivo en un problema ligado a la la sincronización de las diferentes fuentes de dicho archivo. La aplicación desarrollada resuelve este problema utilizando una interfaz con un sistema controlador de versiones permitiendo almacenar toda la información de desarrollo (código fuente) en un solo lugar.


\subsection{Beneficios}

El control de versiones es una herramienta básica para el desarrollo de aplicaciones en grupo. El IDE basado en Web evita la necesidad de instalar clientes de control de versiones en cada máquina, pues el código está bajo un control de versiones centralizado y el manejo de cambios es en su mayor parte transparente para el usuario.

Se disminuye considerablemente la instalación y mantenimiento de las herramientas de programación comúnmente utilizadas como compiladores, editores, etc.

El proceso de compilación es centralizado, requiriendo únicamente un servidor de buen rendimiento y contando incluso con pobres instalaciones o equipos con bajas capacidades de hardware que pasarán a ser clientes de la aplicación.

A nivel personal, el desarrollo de este proyecto permitió afianzar los conocimientos obtenidos durante la carrera, en gran manera en el área de diseño de interfaces y desarrollo de aplicaciones web.


\subsection{Impacto}

\subsubsection{Impacto Económico}

Permite a una empresa de desarrollo de software ahorrar gastos en la compra de herramientas de programación y en la compra de costosos equipos para los programadores, permitiendo la transformación de equipos antiguos en útiles herramientas de trabajo.

\subsubsection{Impacto Social}

El sitio geográfico de trabajo ya no sería un problema, pues como el acceso al IDE puede ser a través de Internet, no habría diferencia al trabajar desde cualquier PC conectado a Internet.

\subsubsection{Impacto Científico y Tecnológico}

El desarrollo de aplicaciones no tiene por que ser un proceso desacoplado del manejo de versiones del código fuente, en la aplicación desarrollada estos 2 elementos se combinan de forma transparente al usuario.

