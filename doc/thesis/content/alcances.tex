\section{ALCANCES}

\subsection{Definición y delimitación del problema}

Un IDE puede convertirse en una aplicación muy compleja si integra muchas herramientas de diversos lenguajes de programación, debido a esto la aplicación desarrollada únicamente soporta las herramientas del lenguaje de programación Java, a pesar de esto, su diseño y arquitectura permiten que sea fácil extenderlo a otras herramientas y lenguajes de programación.

Algunos IDEs incluyen sus propios compiladores y depuradores como es el caso del IDE Eclipse, para resolver el problema de la compilación se creó una interfaz de comunicación con el compilador de la distribución de Java de Sun Microsystems.

El proceso de ejecución se realiza de forma local, lo que significa que los programas generados (los proyectos de desarrollo que realizan los usuarios) no son ejecutados en el ambiente del servidor sino que son transferidos al cliente y este se encarga de su ejecución.

\subsection{Estrategias de desarrollo}

Para el desarrollo de este proyecto se siguió una metología que permitió agilizar los procesos de formulación y diseño del software, disminuyendo la documentación, pero, sin perder claridad sobre el entorno y la funcionalidad interna de la aplicación.

Para apoyar dicho proceso, se usaron herramientas muy útiles para la realización de documentación como son:

\begin{itemize}
	\item El lenguaje de modelado UML \cite{uml}
	\item La herramienta de libre distribución DIA \cite{dia}
	\item El editor de gráficos vectoriales de código abierto Inkscape \cite{inkscape}
	\item El sistema operativo de libre distribución ArchLinux \cite{archlinux}
	\item El sistema manejador de contenido de código abierto Radiant \cite{radiant}
\end{itemize}

