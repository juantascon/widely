\section{RESUMEN}

Un ambiente de desarrollo integrado (IDE) es un marco de trabajo que soporta el desarrollo rápido de aplicaciones de un lenguaje de programación en particular o, algunos casos, el diseño del IDE permite la adaptación para múltiples lenguajes con la ayuda de módulos que se pueden añadir en cualquier momento.

Existen muchos IDEs con excelentes capacidades, la propuesta es un IDE basado en Web, específicamente en el concepto RIA (Rich Internet Application), presentando una ventaja estratégica de poder acceder a la aplicación desde cualquier PC con una configuración típica, esto es, cualquier sistema operativo que incluya un navegador web.

El IDE web pretende combinar de forma eficiente y centralizada la creación, edición, compilación, ejecución, depuración y control de versiones del código fuente de los programas en una herramienta fácil de usar y disponible ya sea a través de una intranet o a través de Internet.

