\section{DEFINICIÓN DEL PROBLEMA}

El problema nace a partir de varias necesidades, lo primero que se intenta resolver es la inexistencia de un sistema de desarrollo disponible a través de una red (Internet o cualquier Intranet) y que se pueda acceder a él utilizando únicamente un navegador web, disminuyendo así la distribución e instalación de software adicional en miles de clientes y con la ventaja de que se pueda utilizar en cualquiera de estos clientes independientemente de la versión del sistema operativo que tengan instalado.

Otro problema que se intenta resolver es la descentralización de la ubicación del código fuente y demás archivos que integran un proyecto de desarrollo de software, que a su vez genera problemas comunes como la pérdida o el desorden de la información o la dificultad de controlar adecuadamente las versiones actuales o pasadas de los archivos.
