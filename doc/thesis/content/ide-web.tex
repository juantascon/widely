\section{IDE WEB}

\subsection{Licencia}

Widely, Acuerdo de Licencia:\newline

Copyright (C) 2006-2007 Juan Diego Tascón V. \newline Este programa es software libre; puede ser redistribuido y/o modificado bajo los términos de la Licencia Publica General GNU GPL versión 2 \cite{gpl} publicada por la Free Software Foundation \cite{fsf}. \newline Este programa es distribuido con la esperanza de que sea útil, pero SIN NINGUNA GARANTÍA; sin siquiera la garantía implicada de MERCADEO o CONVENIENCIA PARA PROPÓSITO PARTICULAR, para mas detalles consulte la Licencia Publica General GNU version 2 incluida en este programa.\newline Si no está incluida una copia de la Licencia Publica General GNU versión 2 \cite{gpl} escriba a la Free Software Foundation \cite{fsf}, Inc., 59 Temple Place - Suite 330, Boston, MA  02111-1307, USA. \newline Las siguientes son dependencias de este programa que no están distribuidos bajo GPL pero que tienen licencias compatibles:

\begin{itemize}
	\item Interprete Ruby: Ruby License \cite{ruby_lic}.
	\item Subversion: Collabnet SubVersion License \cite{svn_lic}.
	\item QooXdoo: GNU Lesser General Public License (LGPL) \cite{lgpl} y Eclipse Public License (EPL) \cite{epl}.
\end{itemize}

\subsection{Metodología}

Mi propuesta para la realización de este proyecto se resume en la declaración de un punto intermedio en donde se describa el problema y los resultados esperados.

El diseño debe ser adaptable, fácil de modificar y su límite de detalle debe ser el punto en donde se pueda escoger, sin que se vea reflejado en un cambio de diseño, cualquier variante de los siguientes factores:

\begin{itemize}
	\item El lenguaje de programación
	\item Las herramientas a utilizar (ej: depuradores, bases de datos, etc)
	\item Las bibliotecas disponibles para el lenguaje de programación
	\item Las tecnologías asociadas (ej: XML, Java, etc)
\end{itemize}

Lo principal de este diseño es diferenciar y describir, pero no detallar, los sistemas que componen la aplicación (ej: sistema de control de usuarios, sistema de envío de la información, etc), luego de esto se procederá con la realización de la aplicación (programación) en donde los detalles de la implementación se describen dentro del propio código (no en diagramas de colaboración o de secuencia), y en donde los resultados sufran un proceso de reingeniería hasta lograr una solución estable que satisfaga los resultados esperados.

Se tendrá siempre en mente el proceso de reingeniría que puede ser iniciado ya sea por cambios en los requerimientos, cambios de plataforma o por errores de integración no previstos, por esta razón el diseño debe estar preparado para que estos cambios afecten al menor número de subsistemas posibles.

Sólo se realizarán los diagramas que sean realmente necesarios y útiles, los cuales serán de dos tipos: los diagramas de división que son aquellos que explican de que forma se separarán los sistemas de la aplicación y los requerimientos del sistema y los diagramas externos que explican procesos separados de la lógica de negocios pero que son necesarios para la culminación del proyecto, para este caso particular se produjeron los siguientes diagramas:

\begin{itemize}
	
	\item Diagramas de División:
	\subitem Casos de Uso
	\subitem Diagrama de Arquitectura
	\subitem Modelo Conceptual
	\subitem Diagrama de Paquetes
	
	\item Diagramas Externos:
	\subitem Modelo de Navegación
	\subitem Diseño de Interfaz
	\subitem Jerarquía de Archivos
	
\end{itemize}

Algunas partes del proyecto que sean más tangibles se realizarán siguiendo un desarrollo conducido por pruebas \cite{tdd}, esto es, primero se harán los casos de pruebas de acuerdo a los requisitos y luego se construirán los sistemas para que pasen estas pruebas de forma satisfactoria.


\subsection{Formulación}
\subsubsection{Casos de Uso}

Los actores involucrados en el proceso del sistema son:

\begin{itemize}

	\item{Usuario Anónimo} (Anonimous User): un usuario anónimo es cualquiera con acceso a la pantalla de inicio del sistema, en teoría cualquier sistema externo (pueden ser personas) que tenga la habilidad de comunicarse con el sistema es considerado un usuario anónimo.
	
	\item{Administrador} (Administrator): el usuario administrador es el que se encarga de configurar las tareas principales del sistema.
	
	\item{Usuario} (User): un usuario es cualquier sistema registrado, validado y aprobado por el administrador.

\end{itemize}

Los casos de uso a desarrollar son:

\begin{description}
	
	\item [Autenticación] (Authetication): \newline
	El sistema debe permitir a un usuario desconocido poder autenticarse con el sistema para que este, dependiendo del proceso de autenticación (modo usuario o modo administrador), lo pueda tratar como un usuario registrado o como un usuario administrador.
	
	\item [Administración de usuarios] (Users Administration): \newline
	El sistema debe permitir al usuario administrador crear y eliminar usuarios registrados, además debe permitir editar los datos de configuración los usuarios.
	
	\item [Cambio de clave de autenticación] (Authentication Key Changing): \newline
	El sistema debe proveer a los usuarios y al administrador un proceso mediante el cual puedan cambiar sus claves de autenticación con el sistema (passwords).
	
	\item [Manejo de archivos y versiones] (File and Versions Handling): \newline
	Gracias a la centralización que provee el IDE Web es posible integrar, de forma inteligente (incluso algunas acciones se pueden hacer de forma transparente al usuario) el sistema de control de versiones con la administración de los archivos del proyecto, a pesar de esta integración, la copia de trabajo de cada usuario estará separada del repositorio del proyecto.
	
	\item [Edición de código fuente] (Source Code Editing): \newline
	El sistema debe tener una interfaz capaz de editar código fuente, opcionalmente se pueden incluir ayudas comunes pero de gran utilidad para un programador como son:
	
	\begin{itemize}
		\item Autoindentación.
		\item Autocompletado de palabras.
		\item Resaltado de sintaxis.
	\end{itemize}
	
	\item [Administración de repositorios y copias de trabajo] (Repositories and Working Copies Administration): \newline
	El programa debe tener una interfaz gráfica que permita crear, acceder y actualizar repositorios de archivos a partir de las copias de trabajo de los usuarios, el manejo del repositorio se hará con una interfaz a un sistema manejador de versiones existente permitiendo aumentar la compatibilidad con otros sistemas, entre los sistemas de control de versiones más populares se encuentran:
	
	\begin{itemize}
		\item SubVersioN
		\item CVS
		\item GIT
		\item Mercurial (HG)
	\end{itemize}
	
	\item [Compilación centralizada] (Centralized Compiling): \newline
	El sistema debe ser capaz de compilar el código fuente de forma centralizada (en el servidor) utilizando interfaces con herramientas de compilación
	
	\item [Envío de Archivos] (File Sending): \newline
	El servidor debe permitir al cliente una interfaz de acceso a sus archivos de forma local, utilizando un sistema de archivos de red para compartir los archivos ejecutables y de código fuente desde el servidor y hacia el cliente, algunos de los más populares sistemas de archivos de red son:
	
	\begin{itemize}
		\item SMB/CIFS
		\item WebDAV
		\item AFS
		\item NFS
		\item NCP
	\end{itemize}
	
\end{description}

\begin{figure}
 \centering
 \includegraphics[scale=1]{../diagrams/output/usecases}
 \caption{Diagrama de Casos de Uso}
 \label{diagrama:casos_uso}
\end{figure}


\subsubsection{Arquitectura}

Al tratarse de una aplicación web la arquitectura que se utilizará para el desarrollo del proyecto sera la arquitectura Cliente-Servidor en la cual el servidor se encarga de centralizar la información y la lógica y la presenta a múltiples clientes encargados de hacer peticiones de acuerdo a sus necesidades.

En el desarrollo de aplicaciones web lo más común es que no sea necesario crear el lado cliente de la aplicación ya que se utiliza como cliente un browser(navegador) incluido en la mayoría de sistemas operativos modernos, por lo tanto, la mayor parte del desarrollo web se centra en la creación del servidor que es encargado, incluso, de generar la interfaz a los clientes para que estos la muestren al usuario final.

En el lado del servidor se utilizará una arquitectura orientada a servicios (SOA \cite{soa}), la cual provee división, modularidad y bajo acoplamiento en la lógica de negocio y en la comunicación interfaz-servidor, estos servicios se desarrollarán utilizando un desarrollo conducido por pruebas \cite{tdd}.\newline El servidor es entonces visto en el lado cliente como un catalogo de servicios, el cliente accede a estos servicios por medio de objetos de acceso a datos (DAOs \cite{dao} ) los cuales son interfaces que representan cada uno de los servicios del catálogo de servicios.

El servidor incluye también un sistema de despacho de peticiones del cliente, el cual permite evacuar de forma inteligente las respuestas pasando por un servidor principal encargado de distribuir las peticiones y luego a través de servidores configurables encargados de atender la petición.

Los compenentes del servidor se integrán utilizando un sistema de declaración de módulos y plugins permitiendo modificar el comportamiento de los diferentes servicios de acuerdo a los plugins cargados en tiempo de ejecución.

\begin{landscape}
\begin{figure}
 \centering
 \includegraphics[scale=0.9]{../diagrams/output/arquitecture}
 \caption{Diagrama de Arquitectura}
 \label{diagrama:arquitectura}
\end{figure}
\end{landscape}


\subsection{Diseño}

\subsubsection{Modelo Conceptual}

Este modelo conceptual describe los elementos reales involucrados en el software, estos son:

\begin{description}
	
	\item [Archivo/Directorio](File/Directory): \newline
	Un archivo es un trozo del código fuente de la aplicación que intenta desarrollar el programador, un directorio es el lugar donde estos archivos se organizan.
	
	\item [Version](Version): \newline
	Una versión se refiere al estado de un archivo en un momento dado y como se relaciona con los cambios históricos hechos sobre este.
	
	\item [Copia de Trabajo](Working Copy): \newline
	Una copia de trabajo es un lugar en donde el programador almacena diversos archivos que pertenecen al proyecto de software, estos archivos pueden ser modificados por el desarrollador produciendo nuevas versiones de los mismos.
	
	\item [Repositorio](Repository): \newline
	Un repositorio es en donde se almacenan los archivos y directorios del proyecto de software, también se lleva un registro histórico de los cambios hechos sobre estos archivos e información adicional sobre los mismos cambios como a que versión corresponde, quien los hizo, cuando, etc.
	
	\item [Compilador/Interprete](Compiler/Interpreter): \newline
	Un compilador es el encargado de generar una aplicación de software a partir de los archivos de código fuente, procesa estos archivos para generar un programa ejecutable, un interprete es similar con la diferencia de que el código es interpretado e inmediatamente ejecutado.
	
	\item [Editor de Texto](Text Editor): \newline
	Un editor de textos es el encargado de asistir al desarrollador en el proceso de visualización y modificación de los archivos de código fuente.
	
	\item [Entorno de Desarrollo](Develop Environment): \newline
	El entorno de desarrollo se refiere a unificación y a la cooperación de diversas herramientas de desarrollo como el editor de textos, el compilador, el manejador de versiones(repositorio - copia de trabajo), etc.
	
	\item [Desarrollador](Developer): \newline
	El desarrollador es cualquier sistema que se comunique con el entorno de desarrollo para producir una aplicación a partir de archivos de código fuente.
	
	\item [Aplicación](Application): \newline
	Una aplicación es una pieza de software que sirve para solucionar un problema especifico.
	
\end{description}

\begin{landscape}
\begin{figure}
 \centering
 \includegraphics[scale=0.9]{../diagrams/output/conceptual}
 \caption{Diagrama Conceptual}
 \label{diagrama:conceptual}
\end{figure}
\end{landscape}


\subsubsection{Modelo de Navegación}

El modelo de navegación muestra el flujo que un usuario final puede seguir, el siguiente es un diagrama de navegación de aplicación web modificado para la adaptación a la técnica de desarrollo AJAX.

El punto inicial de la aplicación es la página de Login, desde ahí tiene dos opciones:

\begin{enumerate}
	
	\item Administración: procede a la página como usuario administrador, aquí se pueden cambiar los parámetros del servidor y crear, eliminar y configurar los usuarios del sistema.
	
	\item IDE: esta es la página principal en donde el usuario desarrolla sus aplicaciones.
	
\end{enumerate}

Como página adicional esta la página de Configuración la cual permite modificar los parámetros de configuración de los usuarios, además de crear y eliminar repositorios y copias de trabajo.

Durante el proceso de interacción entre el cliente y el servidor se utilizan los webservices definidos en el servidor, estos webservices son los encargados de ejecutar los procesos lógicos utilizados durante la navegación del usuario en la aplicación

\begin{landscape}
\begin{figure}
 \centering
 \includegraphics[scale=1]{../diagrams/output/navigation}
 \caption{Diagrama de Navegación}
 \label{diagrama:navegación}
\end{figure}
\end{landscape}


\subsubsection{Diseño de Interfaz}

La interfaz del usuario se divide en 4 pantallas diferentes:

\begin{enumerate}

	\item Login: Es la interfaz en donde el usuario procede a ingresar al sistema.
	
	\item Admin: Es la interfaz de configuración de parámetros del servidor y de administración de usuarios.
	
	\item Config: Es la interfaz de configuración de los parámetros del usuario, así como sus repositorios y sus copias de trabajo.
	
	\item IDE: Es la interfaz principal, el verdadero IDE, contiene un editor, una barra de herramientas, una sistema de consulta de versiones del repositorio y un sistema de modificación de jerarquía de archivos en forma de árbol.

\end{enumerate}

\begin{landscape}
\begin{figure}
 \centering
 \includegraphics[scale=1]{../diagrams/output/ui-login}
 \caption{Diagrama de Interfaz - Login}
 \label{diagrama:ui-login}
\end{figure}
\end{landscape}

\begin{figure}
 \centering
 \includegraphics[scale=1]{../diagrams/output/ui-admin}
 \caption{Diagrama de Interfaz - Admin}
 \label{diagrama:ui-admin}
\end{figure}

\begin{figure}
 \centering
 \includegraphics[scale=0.9]{../diagrams/output/ui-config}
 \caption{Diagrama de Interfaz - Config}
 \label{diagrama:ui-config}
\end{figure}

\begin{landscape}
\begin{figure}
 \centering
 \includegraphics[scale=0.9]{../diagrams/output/ui-ide}
 \caption{Diagrama de Interfaz - IDE}
 \label{diagrama:ui-ide}
\end{figure}
\end{landscape}


\subsubsection{Diagrama de Paquetes}

La arquitectura orientada a servicios permitió dividir la aplicación en paquetes con funcionalidades bien definidas y con bajo acoplamiento, estos paquetes son:

\begin{description}
	
	\item[Addons] (Agregados):\newline
	El módulo de Agregados contiene todos los elementos que no son obligatorios para la correcta ejecución de la aplicación.
	
	\item[Server] (Servidor):\newline
	El módulo del Servidor contiene toda la lógica del negocio distribuida en servicios web así como el sistema de despacho de las peticiones del cliente.
	
	\item[DAOs] (Objetos de Acceso a Datos):\newline
	El módulo de DAOs es el punto de partida del cliente, desde allí se realiza la comunicación con el servidor accediendo a los servicios web que este provee.
	
	\item[GUI] (Interfaz Trafica de Usuario):\newline
	El módulo de Interfaz gráfica provee mecanismos para la interacción con el usuario final de la aplicación, esta se enlaza con el servidor por medio de los DAOs.
	
\end{description}

\begin{landscape}
\begin{figure}
 \centering
 \includegraphics[scale=1]{../diagrams/output/packages}
 \caption{Diagrama de Paquetes}
 \label{diagrama:paquetes}
\end{figure}
\end{landscape}


\subsubsection{Jerarquía de Archivos}

Los componentes de la jerarquía de archivos del proyecto son:

\begin{description}
	
	\item[$<Object\ Directory>$] (Directorio de Objetos):\newline
	Un directorio de objetos es un directorio que contiene subdirectorios en donde cada uno de estos describe un objeto del sistema (ej: cada subdirectorio del directorio $<config\_dir>/users$ describe un objeto tipo usuario en el sistema)
	
	\item[$<data\_dir>$] (Directorio de datos):\newline
	Un directorio de datos es, como su nombre lo indica, un directorio en donde se almacenan datos de trabajo de un objeto del sistema (ej: el directorio $<config\_dir>/users/<user\_id>/data\_dir/wcs/<wc\_name>/data\_dir$ contiene los datos de trabajo de la copia de trabajo $<wc\_name>$ del usuario $<user\_id>$)
	
	\item[$<file.conf>$] (Archivo de configuración):\newline
	Un archivo de configuracion permitiendo tanto a un usuario como al sistema modificar facilmente los datos de configuración de parámetros de los objetos del sistema, posee una sintaxis que es clara y rápida de interpretar por ambos (usuario y sistema)
	
\end{description}

\begin{figure}
 \centering
 \includegraphics[scale=0.9]{../diagrams/output/fs}
 \caption{Diagrama de Jerarquía de Sistema de Archivos}
 \label{diagrama:fs}
\end{figure}


\newpage

\subsection{Implementación}


\subsubsection{Implementación del Software}

La implementación del servidor se encuentra en el mismo directorio en donde se encuentra este documento bajo la siguiente ruta: \textit{\textbf{widely/server/}}

La implementación del cliente se encuentra en el mismo directorio en donde se encuentra este documento bajo la siguiente ruta: \textit{\textbf{widely/gui/}}


\subsubsection{Documentación Interna del Código}

La documentación interna se incluye con el código fuente de la aplicación.


\subsubsection{Manual de Usuario}

El manual de usuario se encuentra en el mismo directorio en donde se encuentra este documento bajo la siguiente ruta: \textit{\textbf{widely/doc/manual/}}


\subsection{Pruebas}

Las pruebas realizadas corresponden a pruebas de integración de cada webservice (la única interfaz lógica entre el servidor y el cliente). La construcción de los webservices siguió un enfoque basado en el desarrollo guiado por pruebas (Test Driven Development \cite{tdd}).

Las pruebas desarrolladas se encuentran en el mismo directorio en donde se encuentra este documento bajo la siguiente ruta:: \textit{\textbf{widely/tests/}}

Los resultados e informes de estas pruebas se encuentran en el mismo directorio en donde se encuentra este documento bajo la siguiente ruta: \textit{\textbf{widely/tests/results/}}
