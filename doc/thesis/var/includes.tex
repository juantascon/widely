%
% Codificacion de archivo UTF-8
%
\usepackage[utf8]{inputenc}

%
% Codificacion de fuente: T1
%
\usepackage[T1]{fontenc}

%
% Idioma del documento: español
%
\usepackage[spanish]{babel}


%
% Para incluir graficos
%
\usepackage{graphicx}

%
% Para tener paginas horizontales dentro de un documento vertical
%
\usepackage{lscape}

%
% Para los enlaces dentro de los PDF
%
\usepackage{hyperref}

%
% Para incluir la bibliografia en el indice
%
\usepackage[nottoc,notlof]{tocbibind}

%
% Utilizar fuentes PostScript
%
%\usepackage{pslatex}

%
% Para colocar mejores pie de pagina
%
\usepackage{fancyhdr}

%
% Espaciado del documento
%
\usepackage{setspace}

%
% La geometria de las paginas
%
\usepackage[right=2cm,left=4cm,top=3cm,bottom=3cm,headsep=0cm,footskip=1cm]{geometry}
