\mode<presentation>
{
	\usetheme{Berkeley}
	\setbeamercovered{transparent}
}

\usepackage[spanish]{babel}
\usepackage[utf8]{inputenc}

\usepackage{times}
\usepackage[T1]{fontenc}

\title{Widely}
\subtitle{Sistema Web de Desarrollo Integrado}

\author{Juan Diego Tascón V.}
\institute{Universidad del Valle \\ Escuela de Ingeniería de Sistemas y Computación }

\date{17 de Septiembre de 2007 }

% If you have a file called "university-logo-filename.xxx", where xxx
% is a graphic format that can be processed by latex or pdflatex,
% resp., then you can add a logo as follows:

% \pgfdeclareimage[height=0.5cm]{university-logo}{university-logo-filename}
% \logo{\pgfuseimage{university-logo}}


\AtBeginSubsection[]
{
	\begin{frame}<beamer>{Contenido}
		\tableofcontents[currentsection,currentsubsection]
	\end{frame}
}

% If you wish to uncover everything in a step-wise fashion, uncomment
% the following command: 

%\beamerdefaultoverlayspecification{<+->}
