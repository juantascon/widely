\section{Conclusiones}

\subsection{Resultado Principal}

\begin{frame}{Resultado Principal:\newline}

Una aplicación web que centraliza en un solo punto los procesos de almacenamiento de información, edición de código, compilación y administración de las versiones de los archivos, facilitando la integración con máquinas clientes de bajas prestaciones y cuyo único requisito es que tenga instalado cualquier sistema operativo moderno.

\end{frame}


\subsection{Resultados Específicos}

\begin{frame}{Resultados Específicos:\newline}

\begin{itemize}
	
	\pause
	\item Administracion de usuarios
	\item Administracion de repositorios y copias de trabajo
	\item Administracion de árbol de archivos
	\item Edición de Codigo Fuente WYSIWYG
	\item Compilación de código Java
	\item Control de versiones
	\item Acceso a datos vía WebDAV
	\item Configuración del sistema de distribución de carga
	
\end{itemize}

\end{frame}


\subsection{Resultados de la Metodología}

\begin{frame}{Resultados de la Metodología:\newline}

\begin{itemize}
	
	\pause \item RUP no fué práctico, generó retrasos en el desarrollo
	\pause \item TDD es el proceso ideal a seguir, pero requiere mucho tiempo
	\pause \item La metodología utilizada funcionó muy bien pero funcionaría en otros tipos de desarrollo?
	
\end{itemize}

\end{frame}
