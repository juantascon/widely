\section{Conclusiones}

\subsection{Resultados}

\begin{frame}{Resultado Principal:\newline}

Una aplicación web que centraliza en un solo punto los procesos de almacenamiento de información, edición de código, compilación y administración de las versiones de los archivos, facilitando la integración con máquinas clientes de bajas prestaciones y cuyo único requisito es que tenga instalado cualquier sistema operativo moderno.

\end{frame}

\begin{frame}{Resultados específicos:\newline}

\begin{itemize}
	
	\item Administracion de usuarios
	\item Administracion de repositorios y copias de trabajo
	\item Administracion de árbol de archivos
	\item Edición de Codigo Fuente WYSIWYG
	\item Compilación de código Java
	\item Control de versiones
	\item Acceso a datos vía WebDAV
	
\end{itemize}

\end{frame}


\begin{frame}{Resultados de la Metodología:\newline}

\begin{itemize}
	
	\item RUP no fué práctico, generó retrasos en el desarrollo
	\pause
	
	\item TDD es el proceso ideal a seguir, pero requiere mucho tiempo
	\pause	
	
	\item La metodología utilizada funcionó muy bien pero funcionaría en otros tipos de desarrollo
	
\end{itemize}

\end{frame}


\subsection{Opiniones}

\begin{frame}{Opiniones Personales:\newline}

\begin{itemize}
	
	\item Ruby es el mejor lenguaje que existe
	\item Windows solo sirve para 3 cosas
	\item Deberian eliminar a Gnome y a todos los que lo usan
	\item Novell y Linspire son unos vendidos
	
\end{itemize}

\end{frame}
