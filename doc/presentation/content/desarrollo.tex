\section{Proceso de Desarrollo}


\subsection{Metodología}

\begin{frame}{Metodología Utilizada:\newline}

La metodología utilizada se diseño pensando en corregir falencias de otras metodologías como RUP, XP y TDD, estas a su vez aportan excelentes conceptos y puntos de vista que se utilizaron en la metodología. 

\end{frame}

\begin{frame}{Bases de la Metodología:\newline}

\begin{itemize}
	
	\item Diseño Basado en Reingeniería
	\pause
	\item Diagramas de División y Diagramas Externos
	\pause
	\item Detalles de Implementación en Documentación Interna
	\pause
	\item Desarrollo Conducido por Pruebas
	
\end{itemize}

\end{frame}


\subsection{Tecnologías Utilizadas}

\begin{frame}{Ruby:\newline}

Ruby es un lenguaje de programación orientada a objetos e interpretado, creado por Yukihiro Matsumoto, que tiene una sintaxis inspirada en varios lenguajes de programación como Perl, Smalltalk, Python, C, Lisp, entre otros.

\end{frame}

\begin{frame}{Principales Caracteristicas de Ruby:\newline}

\begin{itemize}
	
	\item Orientado a Objetos
	\pause
	\item Duck Typing
	\pause
	\item Closures
	\pause
	\item Reflexión
	\pause
	\item DRY (Don't Repeat Yourself)
	
\end{itemize}

\end{frame}

\begin{frame}{Otras Tecnologías Utilizadas:\newline}
\begin{itemize}
	
	\item SOA - DAO
	\pause
	\item QooXdoo
	\pause
	\item JSON
	\pause
	\item WebDAV
	\pause
	\item Pound
	\pause
	\item YML
	
\end{itemize}
\end{frame}
