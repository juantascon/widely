\section{Introducción Al Problema}


\subsection{Conceptos}

\begin{frame}{Conceptos Básicos:\newline}

\begin{itemize}
	
	\pause \item IDE: Integrated Development Environment (Entorno de Desarrollo Integrado)
	%es un programa compuesto por un conjunto de herramientas integradas de forma eficiente para facilitar el trabajo de un desarrollador.
	
	\pause \item RIA: Rich Internet Applications (Aplicaciones Ricas de Internet)
	%son aplicaciones web que tienen funcionalidades propias de las tradicionales aplicaciones de escritorio, generalmente la interfaz gráfica se ejecuta en un navegador web mientras que la lógica y el procesamiento de datos se ejecuta en el servidor de la aplicación.
	
	\pause \item Control de Versiones
	%El control de versiones se refiere al proceso de administrar múltiples revisiones de la misma unidad de información, surge en el proceso de desarrollo de software debido a su naturaleza de tener múltiples programadores trabajando sobre múltiples archivos simultáneamente.
	
\end{itemize}

\end{frame}


\subsection{El Problema}

\begin{frame}{Definición del Problema:\newline}

Diseñar y construir una aplicación web IDE basada en el concepto RIA que facilite el proceso de edición, compilación, ejecución y control de versiones de un proyecto de software.

\end{frame}
